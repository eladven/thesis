\documentclass[landscape,a0,final]{a0poster}
%%% Option "a4resizeable" makes it possible ot resize the
%   poster by the command: psresize -pa4 poster.ps poster-a4.ps
%   For final printing, please remove option "a4resizeable" !!

\usepackage{epsfig}
\usepackage{color}
\usepackage{shadow}
\usepackage{multicol}
\usepackage{pstricks}

%%% Size of Poster etc.
\newlength{\textArea}
\setlength{\textArea}{108cm}  % for a0b = A0-big !!
\setlength{\columnsep}{3cm}
\setlength{\columnseprule}{0.5mm}
\setlength{\parindent}{0.0cm}


%%% \mysection - replacement for \section
% with colored section-title and automatic numbering
%\newcounter{section}
\setcounter{section}{1}
\newcommand{\mysection}[1]{
  \vspace{1cm}
  \begin{center}
    \shabox{\colorbox{mainCol}{
      \begin{minipage}[c][50pt][c]{0.9\columnwidth}
        \vspace{2.5mm}
        \begin{center}
          {\bf\textcolor{TextCol}{\arabic{section}.~#1}}
        \end{center}
        \vfill
      \end{minipage}
    }}
  \end{center}
  \par\vspace{0.75cm}
  \stepcounter{section}
}

%%% \myplainsection - replacement for \section
\newcommand{\myplainsection}[1]{
  \vspace{2cm}
  {\bf\textcolor{TextCol}{\arabic{section}.~#1}}
  \par\vspace{0.75cm}
  \stepcounter{section}
}

%%% \mysubsection - replacement for \subsection
\newcommand{\mysubsection}[1]{
  \vspace{1cm}
  {\bf\textcolor{TextCol}{#1}}
  \par\vspace{0.75cm}
}

%%% \myfig - replacement for \figure
% necessary, since in multicol-environment \figure won't work
\newcommand{\myfig}[3][0]{
\begin{center}
  \vspace{0.5cm}
  \scalebox{0.75}{\includegraphics[width=#3\hsize,angle=#1]{#2}}
  \nobreak\medskip
\end{center}}

%%% \mycaption - replacement for \caption
% necessary, since in multicol-environment \figure and
% therefore \caption won't work
%\newcounter{figure}
\setcounter{figure}{1}
\newcommand{\mycaption}[1]{
  \vspace{0.8cm}\par
  {\renewcommand{\baselinestretch}{1}\small
   \setbox0=\hbox{\sc Figure \arabic{figure}: \sl #1}
   \ifdim\wd0>\hsize
     {\sc Figure \arabic{figure}: \sl #1}
   \else
     \centerline{\box0}
   \fi
   \par
   \stepcounter{figure}
  }
}




\usepackage{array}
\newcommand{\arraylines}[1]{{\newcolumntype{r}{c|}#1}}
\newcommand{\arraydashlines}[1]{{\newcolumntype{l}{c:}#1}}
\usepackage{arydshln}

%#1 Size of minipage 1
%#2 Contents of partition 1
%#3 Size of minipage 2
%#4 Contents of partition 2
% \textwidth is less than 125mm
\newcommand{\bp}[4]{%
\begin{minipage}{16.1cm}
  \begin{minipage}{#1}
  #2
  \end{minipage} \hfill \begin{minipage}{#3}
  #4
 \end{minipage}
\end{minipage}
}

%background color
%\special{!userdict begin /bop-hook{gsave clippath 0.98 0.92 0.73 setrgbcolor fill grestore}def end}
\special{!userdict begin /bop-hook{gsave clippath 0.99 0.93 0.84 setrgbcolor fill grestore}def end} %light yellow
%\special{!userdict begin /bop-hook{gsave clippath 1 0.8 0.93 setrgbcolor fill grestore}def end} %


%%%%%%%%%%%%%%%%%%%%%%%%%%%%%%%%%%%%%%%%%%%%%%%%%%%%%%%%%%%%%%%%%%%%%%%%%%%%%%%%
%%% Begin of Document

\begin{document}

%%%%%%%%%%%%%%%%%%%%%%%%%%%%%%%%%%%%%%%%
% Definition of Colors
% Background- and Text-color
\definecolor{mainCol}{rgb}{0.91,0.91,1}
\definecolor{TextCol}{rgb}{0,0,0}

\setlength{\sboxrule}{2pt}
\setlength{\sboxsep}{0pt}

\vspace*{0.5cm}

%%% Topic and Posternumber
\shabox{\colorbox{mainCol}{
  \begin{minipage}[c][0.1\textArea][c]{0.1\textArea}
    \begin{center}
      \myfig{ic_logo.eps}{1}
%      {\LARGE Imperial College}\\
%      \vspace{1cm}
%      {\LARGE London}
    \end{center}
  \end{minipage}}}\hfill
%%% Title und Author
\shabox{\colorbox{mainCol}{
  \begin{minipage}[c][0.1\textArea][c]{0.78\textArea}
\begin{center}
{\vspace{1.0cm}\Huge Control of Grid-Connected DC-AC Power Converters}\\
\vspace{1.0cm}
{\Large {\sc Qing-Chang Zhong, Tim Green, Jun Liang and George Weiss}\\
\vspace{0.7cm}
Control and Power Group\\
Department of Electrical and Electronic Engineering\\
Imperial College London
%\vspace{1cm}
%\textit{g.weiss@ic.ac.uk}
}
\end{center}
\end{minipage}}}\hfill
%%% University-emblem
\shabox{\colorbox{mainCol}{
  \begin{minipage}[c][0.1\textArea][c]{0.1\textArea}
    \begin{center}
      \myfig{epsrc_logo.eps}{1}
    \end{center}
  \end{minipage}}}\vspace{1.5cm}


%%% Begin of Multicols-Enviroment
\begin{multicols}{6}

%%% Abstract

\begin{center}
%\vspace{1cm}
{\large{\bf\textcolor{TextCol}{Abstract}}}
\end{center}
\par
%\vspace{1.5cm}

 Electric utilities have historically satisfied customer demand by generating
 electricity centrally and distributing it through an extensive transmission
 and distribution network.
 For various reasons, an alternative approach under consideration is to
 generate part of the electricity locally in small generating units.
 This is called distributed generation (DG).
 Most of the DG technologies require DC-AC power converters to interface
 with the local grid.
 This research explores the control problems existing in such grid-connected converters.
 These problems include voltage control, power control and neutral point control.
 Repetitive control is used in the voltage controller to enhance the tracking
 performance of the converter with a sinusoidal reference voltage, which
 is generated by the power controller, so that the local consumer receives
 a clean sinusoidal voltage.
 The voltage controller is designed using the solution of the standard $ H^{\infty } $ control problem.
 Decoupling the effect of the required active power and reactive power on
 the reference voltage, a power controller is designed to generate a reference voltage in 
 a small range around the nominal value, guaranteeing the current flowing to the grid within a given limited range.
 The neutral point control problem is to maintain the two contacts of the
 DC source at equal voltages (of opposite sign) with respect to the neutral,
 even in the presence of an unbalanced load.
 A circuit topology for the neutral point is proposed.
 After modeling this topology, an $ H^{\infty } $ controller is designed, taking into account the rank condition and the
 selection of weighting functions etc.
 Simulations are given and experiements are being undertaken.

\vspace{5mm}
%%% Introduction
\mysection{System Description}

%\myfig{distrib_gen.eps}{0.8}

\myfig{blockdiagram.eps}{1.0}
%\vspace{5mm}

%\myfig{controlstructure.eps}{0.8}

Control problems involved in this project:
\begin{itemize}
\item voltage control: \( V_{c} \) as close to $V_{ref}$ as possible
\item power control: to regulate the active power and the reactive power injected into
the grid
\item neutral point control: the neutral point should not be drifting with respect to the DC link
\item phase-locked loop (PLL): to synchronize the converter with the grid
\end{itemize}

\mysection{Voltage Control}

Single-phase circuit

\myfig{singlephaseelec.eps}{1.2} 

A robust repetitive control scheme
\myfig{../../work/RobustRC/contstruc_v_c.eps}{0.7} 

Formulation into an $H^\infty$ problem
\myfig{../../work/RobustRC/struct_RRC_c.eps}{0.8} 

\mysection{Power Control}
\mysubsection{Decoupling and reference voltage}
\myfig{vcphasor.eps}{0.85}

Assume that \( \bar{V}_{g}=V_{g}\angle 0 \) and the current to the
grid is \( \bar{I}_{g} \), as shown in the above phasor diagram, then the active power and the reactive power injected
into the grid are, respectively,
\[
P=I_{gx}V_{g}\textrm{ and }Q=-I_{gy}V_{g}.\]

The output voltage is \begin{eqnarray*}
 & \bar{V}_{c} & =\bar{V}_{g}+\bar{I}_{g}Z_{g}\angle \theta _{zg}\\
 &  & =\bar{V}_{g}+I_{gx}Z_{g}\angle \theta _{zg}+jI_{gy}Z_{g}\angle \theta _{zg}.
\end{eqnarray*}

Defining the following static models:

\[V_{P}\doteq \frac{Z_{g}}{V_{g}}P \textrm{  and  } V_{Q}\doteq \frac{Z_{g}}{V_{g}}Q,\]

then the voltage reference is given by\[
\bar{V}_{ref}\approx \bar{V}_{c}=\bar{V}_{g}+V_{P}\angle \theta _{zg}+V_{Q}\angle (\theta _{zg}-\frac{\pi }{2}).\]
The corresponding control structure, as if two decoupled sub-systems, is

\myfig{PQcontrol.eps}{0.8} 

Ways to obtain \( P_{ref} \), \( Q_{ref} \):
\begin{itemize}
\item using the reference values provided by the network manager;
\item set \( Q_{ref}=0 \) (to minimize current \( I_{g} \));
\item set \( Q_{ref} \) such that the amplitude of the voltage reference
is in a small range 
\end{itemize}

\mysubsection{Voltage constraint}
\myfig{PQsmalltolerence.eps}{0.8} 

\[ A_{x}=V_{g}+V_{P}\cos \theta _{zg} \]
\[ A_{y}=V_{P}\sin \theta _{zg} \]

Amplitude equation (when \( A \) is not inside the band):\[
(A_{x}+V_{Q}\cos \theta _{A})^{2}+(A_{y}+V_{Q}\sin \theta _{A})^{2}=V^{2}_{gn}(1+\delta )^{2}\]
where \( \theta _{A}=\left\{ \begin{array}{c}
\theta _{zg}+\frac{\pi }{2}\qquad (A\textrm{ is outer},\textrm{ }\delta >0)\\
\theta _{zg}-\frac{\pi }{2}\qquad (A\textrm{ is inner},\textrm{ }\delta <0)
\end{array}\right. . \)

The corresponding \( V_{Q} \) (with minimal absolute
value) is
\[
V_{Q}=V_{g}\sin \theta _{zg}(\sqrt{1-\frac{A^{2}_{x}+A^{2}_{y}-V^{2}_{gn}(1+\delta )^{2}}{(V_{g}\sin \theta _{zg})^{2}}}-1)\]
and the reference reactive power is \[ Q_{ref}=\frac{V_{g}}{Z_{g}}V_{Q}. \]

If \( A \) is on the outer side of the band, then \( V_{Q}<0 \)
(\( Q_{ref}<0 \)); if \( A \) is on the inner side of the band,
then \( V_{Q}>0 \) (\( Q_{_{ref}}>0 \)).

If \( A \) is inside the band, then \( V_{Q}=0 \) (\( Q_{ref}=0 \)). 

\mysubsection{Current constraint}

The current flowing to the grid can be limited by the circle in the figure below.
\myfig{PQcontrol_Imsmall.eps}{0.8} 

\begin{itemize}
\item \( \theta _{zg} \) should be as close to \( \frac{\pi }{2} \) as possible;
\item \( V_{Qm}=\frac{Z_{g}}{V_{g}}\sqrt{V^{2}_{g}I_{gm}^{2}-P^{2}} \)
\item the minimum \( \delta  \) can be found for a given \( P \);
\item the relationship between \( P_{max} \) and \( V_{g} \) can be found.
\end{itemize}

%\vspace{5mm}
\mysection{Neutral Point Control}
%\vspace{5mm}

\begin{itemize}
\item unbalanced loads need a current path and a stable neutral point

\begin{itemize}
\item neutral point shifting
\item unbalanced or modulated output voltage
\item the presence of zero-sequence voltage or DC component 
\item larger neutral current
\item common-mode voltage
\end{itemize}
\item multi-level converters need a stable neutral point
\end{itemize}

%\vspace{5mm}
\mysubsection{Existing schemes}

%\vspace{5mm}
\bp{6.5cm}{
\myfig{../../work/NeutralLeg/splitDC_c.eps}{1.2}
\centering Split DC link
}{9.2cm}{
\myfig{../../work/NeutralLeg/conventional4leg_c.eps}{1.2} 
Conventional Neutral leg
}

%\vspace{5mm}
\mysubsection{The proposed scheme}

\myfig{../../work/NeutralLeg/circuit_2CR_c.eps}{0.9}

\[
V_{DC}=V_{+}-V_{-} \;  \textrm{and}  \; V_{ave}=\frac{V_{+}+V_{-}}{2},\]
\[
V_{0}=-\frac{C_{N1}-C_{N2}}{C_{N1}+C_{N2}}\frac{V_{DC}}{2},\]

\[
i_{c}=(C_{N1}+C_{N2})\frac{d(V_{ave}-V_{0})}{dt},\]
\[
u_{N}=\frac{p}{2}V_{DC}+V_{ave},\]
where $p$ is the average voltage of the IGBT firing pulse (amplitude: $ \pm 1$ ).

\myfig{../../work/NeutralLeg/blkd_neuL_V0.eps}{0.8} 

Formulation into an $ H_{\infty }$ control problem

\myfig{../../work/NeutralLeg/controlBlk_Hinf_c.eps}{1} 
\[
\left[ \begin{array}{c}
z\\
y
\end{array}\right] =\mathbf{P}\left[ \begin{array}{c}
w\\
u
\end{array}\right] ,\qquad u=\mathbf{K}\: y,\]
where
\arraylines{\small\arraydashlines{
\[
\mathbf{P}=\left[ \begin{array}{ccrlc}
A & 0 & 0 & B_{1} & B_{2}\\
B_{v}C_{1a} & A_{v} & 0 & B_{v}D_{11a} & B_{v}D_{12a}\\
B_{u}C_{1b} & 0 & A_{u} & B_{u}D_{11b} & B_{u}D_{12b}\\
\hline D_{v}C_{1a} & C_{v} & 0 & D_{v}D_{11a} & D_{v}D_{12a}\\
D_{u}C_{1b} & 0 & C_{u} & D_{u}D_{11b} & D_{u}D_{12b}\\
\hdashline C_{2} & 0 & 0 & D_{21} & D_{22}
\end{array}\right] ,\]
}with\[
A=\left[ \begin{array}{cc}
0 & \frac{1}{L_{N}}\\
-\frac{1}{C_{N1}+C_{N2}} & 0
\end{array}\right] ,\: B_{1}=\left[ \begin{array}{cc}
0 & \frac{\rho }{L_{N}}\\
\frac{1}{C_{N1}+C_{N2}} & 0
\end{array}\right] ,\: B_{2}=\left[ \begin{array}{c}
\frac{V_{DC}}{2L_{N}}\\
0
\end{array}\right]\]
 \[
C_{2}=\left[ \begin{array}{cc}
-1 & 0\\
0 & 1
\end{array}\right] ,\: D_{21}=\left[ \begin{array}{cc}
1 & 0\\
0 & \rho 
\end{array}\right] ,\: D_{22}=\left[ \begin{array}{c}
0\\
0
\end{array}\right] ,\]
\[
C_{1a}=\left[ \begin{array}{cc}
0 & 1
\end{array}\right] ,\: D_{11a}=\left[ \begin{array}{cc}
0 & \rho 
\end{array}\right] ,\: D_{12a}=\left[ \begin{array}{c}
0
\end{array}\right] ,\]
\[
C_{1b}=\left[ \begin{array}{cc}
0 & 1
\end{array}\right] ,\: D_{11b}=\left[ \begin{array}{cc}
0 & \rho 
\end{array}\right] ,\: D_{12b}=\left[ \begin{array}{c}
\frac{V_{DC}}{2}
\end{array}\right] .\]


\mysubsection{Weighting functions}
\[
W_{u}(s)=\left[ \begin{array}{rc}
A_u & B_u\\
\hline C_u & D_u
\end{array}\right] =\frac{k}{\omega _{i}L_{N}}\frac{s+\omega _{i}}{s+\omega _{l}},\]
\[
W_{v}(s)=\left[ \begin{array}{rc}
A_v & B_v\\
\hline C_v & D_v
\end{array}\right]=\frac{g\omega _{h}}{\omega _{i}}\frac{(s+\omega _{o})(s+\omega _{i})}{(s+\omega _{l})(s+\omega _{h})}.\]
}%for arraylines


\mysubsection{A design example}

Neutral leg parameters: \( L_{N}=10 \)mH, \( C_{N1}=C_{N2}=4000\mu  \)F, \( V_{DC}=900 \)V. 

Tuning parameters: \( k=0.1 \), \( g=10 \),\hfill \( \rho =0.0001 \)
and \( \delta _{1}=\delta _{2}=10^{-6} \). 

Weighting parameters: \( \omega _{l}=0.001 \)rad/sec, \( \omega _{h}=10^{8} \)rad/sec, \hfill
\( \omega _{o}=1500 \)rad/sec \hfill and \( \omega _{i}=10000 \)rad/sec. 

Using MATLAB\texttrademark \( \mu  \)-analysis toolbox, the \( H^{\infty } \)
controller \( \mathbf{K}=\left[ \begin{array}{cc}
\mathbf{K}_{i} & \mathbf{K}_{v}
\end{array}\right]  \) is obtained as

\scalebox{0.57}{\(
\mathbf{K}=\left[ \begin{array}{c}
\frac{2.2222e-009(s+1.59e012)(s+9.97e007)(s+0.001302)(s^{2}+0.0007977s+1.957e-007)}{(s+1e008)(s+2.392e004)(s+1.25)(s+0.06372)(s+0.001)}\\
\\
\frac{-0.0022222(s+4.906e009)(s-3.057e005)(s+1.249)(s+0.001)(s-1.585e-006)}{(s+1e008)(s+2.392e004)(s+1.25)(s+0.06372)(s+0.001)}
\end{array}\right] ^{T}.\)}

\mysubsection{Controller reduction}
The obtained controller is not realistic due to the high-frequency modes. It has to be reduced, e.g. according to the following rules:
\begin{itemize}
\item The mode resulted from \( \omega _{l} \) should be changed back to
\( s=0 \);
\item The small zero with comparison to \( \omega _{l} \) should be changed
back to \( s=0 \);
\item The zeros and poles very close to each other should be canceled;
\item Any zero or pole which corresponds to a corner frequency larger
than \( \omega _{i} \) (here, \( 10^{4} \)rad/sec) are changed by
omitting the corresponding \( s \) in the transfer function, i.e.,
substituted by a proportional gain;
\item No differentiator in the controller due to the requirement of internal
stability.
\end{itemize}
The reduced controller is{\small\[
\mathbf{K}=\left[ \begin{array}{cc}
0.1473\frac{(s+0.001302)(s+0.0007977)}{(s+1.25)(s+0.06372)} & 1.3933\frac{s-1.585e-006}{s+0.06372}
\end{array}\right] .\]}
It can be further reduced to a P controller\[
\mathbf{K}_{p}=\left[ \begin{array}{cc}
0.1473 & 1.3933
\end{array}\right] .\]

\bp{7.8cm}{\myfig{../../work/NeutralLeg/bode_Hinf_15_c.eps}{1.2} 
}{7.8cm}{\myfig{../../work/NeutralLeg/bode_T_Hinf_15_c.eps}{1.2}
}
\bp{7.8cm}{Bode plots of the controller
}{7.8cm}{Bode plots of transfer functions from $i_N$ to $y$
}

\mysection{Simulation Results}

\mysubsection{Voltage control}

\bp{7.8cm}{\myfig{../../work/RobustRC/Vcerror_c.eps}{1.2} 
}{7.8cm}{\myfig{../../work/RobustRC/Vg_c.eps}{1.2}
}

\mysubsection{Neutral point control}

A single-phase buck converter is used to simulate the neutral
current \( i_{N} \) of a three-phase converter. The fundamental frequency
of this buck converter can be set at different frequencies to simulate
different harmonic components in a neutral line. Here, \( f_{s}=10 \)kHz and the neutral current 
is about \( 68 \)A peak at \( 50 \)Hz. The obtained $V_{ave}$  is less than $0.4$V.

\bp{7.8cm}{
\myfig{../../work/NeutralLeg/50eILINHinf15_c.eps}{1.2}
}{7.8cm}{
\myfig{../../work/NeutralLeg/50uNIcHinf15_c.eps}{1.2} 
}
The simulation results of power control are omitted.
%%% Figures:
%\begin{center}
  % first argument: eps-file
  % second argument: stretching-factor relative to Column-width (<1)
  % optional argument: rotation angle (0-360), default=0
  %\myfig[90]{}{0.5}
  %\mycaption{Emblem of the University of Regensburg (rotated by $90^\circ$).}
%\end{center}

{\bf\large Acknowledgement}

This project is supported by the EPSRC (Grant No. GR/N38190/1).

%%% References
\bibliographystyle{alpha}
\bibliography{poster-example}


\end{multicols}

\end{document}
