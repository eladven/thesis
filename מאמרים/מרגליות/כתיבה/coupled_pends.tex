

% \documentclass[10pt,journal,onecolumn ]{IEEEtran}
 % \documentclass[final,letter ]{IEEEtran}
% \documentclass[10pt,twocolumn ]{IEEEtran}
%
 \documentclass[10pt,onecolumn,twoside,letter]{IEEEtran}

% If IEEEtran.cls has not been installed into the LaTeX system files,
% manually specify the path to it like:
% \documentclass[12pt,journal,compsoc]{../sty/IEEEtran}

\usepackage{url}
%\usepackage{natbib}
\usepackage{amsmath}
\usepackage{amsfonts}
\usepackage{theorem}
\usepackage[dvips]{epsfig}      % ******* for figures ***************
\usepackage{graphicx}
\usepackage{enumerate}

\usepackage{verbatim}
\usepackage{amssymb}
\usepackage{amsbsy}
\usepackage{setspace}
\usepackage{url}

\theoremstyle{plain}
\newtheorem{Theorem}{Theorem}
\newtheorem{Lemma}{Lemma}
\newtheorem{Proposition}{Proposition}
\newtheorem{Corollary}{Corollary}
\newtheorem{Conjecture}{Conjecture}
\newtheorem{Problem}{Problem}

 \newcommand{\Ave}{\operatorname{Ave}}
 \newcommand{\Int}{\operatorname{Int}}
 \newcommand{\real}{\operatorname{Re}}

{\theorembodyfont{\rmfamily} \newtheorem{Remark}{Remark}}
{\theorembodyfont{\rmfamily} \newtheorem{Assumption}{Assumption}}
{\theorembodyfont{\rmfamily} \newtheorem{Example}{Example}}
{\theorembodyfont{\rmfamily} \newtheorem{Definition}{Definition}}
{\theorembodyfont{\rmfamily} \newtheorem{Question}{Question}}

\newcommand {\R}{\mathbb R}
\newcommand {\A}{\mathbb A}

\newcommand{\be}{\begin{equation}}
\newcommand{\ee}{\end{equation}}
\newcommand{\sgn}{\operatorname{{\mathrm sgn}}}
%%%%%%%%\newcommand{\ve}[1]{\mbox{\boldmath${#1}$}}
\newcommand{\V}{\mathcal V}
\newcommand{\U}{\mathcal U}
\newcommand{\N}{\mathbb N_0}
\newcommand{\B}{\mathcal B \mathcal B}
\newcommand{\W}{\mathcal W}

\newcommand{\sname}{} \newcommand{\slabel}[1]{\debug{\fbox{\tiny \sname #1}}\label{\sname #1}}
\newcommand{\debug}[1]{}              % final version
\newcommand{\FB}{\begin{figure}[t]\centering} \newcommand{\FE}[2]{\caption{#2 \debug{\fbox{\sname #1}}} \slabel{#1} \end{figure}} \newcommand{\tB}{\begin{table}[hbtp]\centering}
\newcommand{\tE}[2]{\caption{#2 \debug{\fbox{\sname #1}}}\slabel{#1} \end{table}} \newcommand{\FIG}[3]{\FB\finput{#1}\FE{#2}{#3}}
\newcommand{\finput}[1]{\input{#1}}   % input figures
%%%% Example: \FIG{fig1}{1-fig}{The Multilink environment}

% counter used by the single column equation
\newcounter{mytempeqncnt}

% correct bad hyphenation here
%\hyphenation{op-tical net-works semi-conduc-tor}


\begin{document}
%
 \title{Synchronization of Coupled Pendulums\thanks{This research is partially supported by a   research grant
from  the  Israeli Ministry of ....}}


%%\begin{comment}
%%%%%%%%
\author{Elad Venezian  and Michael Margaliot \IEEEcompsocitemizethanks{\IEEEcompsocthanksitem
E. Venezian is with the School of Electrical Engineering, Tel-Aviv
University, Tel-Aviv 69978, Israel.
E-mail: ravehalon@gmail.com \protect\\
M. Margaliot is with the School of Electrical Engineering and the Sagol School of Neuroscience, Tel-Aviv
University, Tel-Aviv 69978, Israel.
E-mail: michaelm@eng.tau.ac.il \protect\\
}}
%\end{comment}


\maketitle


\begin{abstract}
%%%%

%%%%
 %%
\end{abstract}

\begin{IEEEkeywords}
%%

%%%
\end{IEEEkeywords}




%% Introduction
%%%%%%%%%
\section{Introduction}
%%%%%%%%%%%%%%%%%%%%%%%%%%%%%%%%%%%




\section{The  model}\label{sec:model}
%%%%%%%%%%%%%%%
Consider a nonlinear pendulum with forcing~$u$:
\[
            \ddot x+\alpha \dot  x+\sin(x)=u ,
\]
where~$x$ is the angle,dfdf and~$\alpha>0$.

Define~$\dot{y}:=\dot x+\frac{\alpha}{2}x$. Then
\[
            \dot y =-\sin(x)-\frac{\alpha}{2}\dot x +u.
\]
  Thus,
\begin{align*}
%%%
                \begin{bmatrix} \dot x \\ \dot y \end{bmatrix} = \begin{bmatrix} -\frac{\alpha}{2}x+y \\  \frac{\alpha^2}{4}x-\sin(x)-\frac{\alpha}{2} y \end{bmatrix}
                +\begin{bmatrix} 0 \\u \end{bmatrix}.
%%%
\end{align*}
The Jacobian of this dynamics is
\[
            J=\begin{bmatrix}   -\frac{\alpha}{ 2} & 1 \\\frac{\alpha^2}{4}-\cos(x) &-\frac{\alpha}{2}       
               \end{bmatrix},
\]
and its symmetric part is 
\[
            J_s:=\frac{J+J'}{2}= \begin{bmatrix}   -\frac{\alpha}{ 2} & \frac{  1-\cos(x)+\frac{\alpha^2}{4}}{2}\\
           \frac{  1-\cos(x)+\frac{\alpha^2}{4}}{2} &-\frac{\alpha}{2}
               \end{bmatrix}.
\]
%%%%%%%%%%%%%%%%%%%%%%%%%%%%%%%%%%%%%%%%%%
%%%%%%%%%%%%%%%%%%%%%%%%%%%%%%%%%%%%%%%%%
The eigenvalues of~$J_s$ are
\[
             \frac{1}{2} \left(  -  {\alpha}  \pm   |  1-\cos(x)+\frac{\alpha^2}{4}  | \right)=\frac{1}{2} \left(  -  {\alpha}  \pm    (  1-\cos(x)+\frac{\alpha^2}{4} )   \right),
\]
so
\[
            \lambda_{\max}(J_s)=\frac{1}{2} \left(   (\frac{\alpha}{2}-1)^2 -\cos(x)     \right).
\]
Let~$q \in[0,\pi/2]$ satisfy~$\cos(q)=(\frac{\alpha}{2}-1)^2$.
(THIS MEANS THAT WE NEED ABOUND ON ALPHA, NO?)
Then~$ \lambda_{\max}(J_s)<0$ for all~$x\in(-q,q)$.
In particular, for~$\alpha=2$, we have that~$ \lambda_{\max}(J_s)<0$ for all~$x\in(-\pi/2,\pi/2)$.

 
Recall that for the Euclidean vector norm, the induced
matrix norm is~$|A|=(\lambda_{\max}(A'A))^{1/2} $,
 and the induced matrix measure is~$\mu(A)=\lambda_{\max}( \frac{A+A'}{2} )$ (see, e.g.,~\cite{vid}).
Standard arguments from contraction theory (see, e.g.,~\cite{LOHMILLER1998683,sontag_contraction_tutorial}) imply
that   trajectories that remain in the closed  region~$x \in [-q-\varepsilon,q+\varepsilon]$, with~$\varepsilon>0$,
contract with respect to the Euclidean  vector  norm. 



\bibliographystyle{IEEEtranS}
\bibliography{RFM_bibl}



\end{document}
