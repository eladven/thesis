%% LyX 2.2.0 created this file.  For more info, see http://www.lyx.org/.
%% Do not edit unless you really know what you are doing.
\documentclass[12pt,english]{report}
\usepackage{mathptmx}
\renewcommand{\familydefault}{\rmdefault}
\usepackage[T1]{fontenc}
\usepackage[latin9]{inputenc}
\usepackage[a4paper]{geometry}
\geometry{verbose,tmargin=2cm,bmargin=2cm,lmargin=2cm,rmargin=2cm,headheight=1cm,headsep=1cm,footskip=1cm}
\setcounter{secnumdepth}{3}
\setcounter{tocdepth}{3}
\setlength{\parskip}{\medskipamount}
\setlength{\parindent}{0pt}
\usepackage{babel}
\usepackage{verbatim}
\usepackage{float}
\usepackage{mathtools}
\usepackage{amsmath}
\usepackage{amssymb}
\usepackage{graphicx}
\usepackage{setspace}
\usepackage{esint}
\usepackage[numbers]{natbib}
\PassOptionsToPackage{normalem}{ulem}
\usepackage{ulem}
\usepackage{nomencl}
% the following is useful when we have the old nomencl.sty package
\providecommand{\printnomenclature}{\printglossary}
\providecommand{\makenomenclature}{\makeglossary}
\makenomenclature
\doublespacing
\usepackage[unicode=true,pdfusetitle,
 bookmarks=true,bookmarksnumbered=false,bookmarksopen=false,
 breaklinks=false,pdfborder={0 0 1},backref=false,colorlinks=false]
 {hyperref}
\usepackage{breakurl}

\makeatletter

%%%%%%%%%%%%%%%%%%%%%%%%%%%%%% LyX specific LaTeX commands.
%% Because html converters don't know tabularnewline
\providecommand{\tabularnewline}{\\}

%%%%%%%%%%%%%%%%%%%%%%%%%%%%%% User specified LaTeX commands.
\usepackage{tauthesis}
\usepackage[font={small,bf}, labelfont={small,bf}, margin=1cm]{caption}
\usepackage{titlesec}
\newcommand{\hsp}{\hspace{20pt}}
\titleformat{\chapter}[hang]{\Huge\bfseries}{\thechapter\hsp}{0pt}{\Huge\bfseries}
\usepackage{tikz}
\usepackage[europeanresistors,americaninductors]{circuitikz}
 
\Title{\textbf{Stability of Synchronous Generators}}
\Author{\textbf{\large Elad Venezian}}
\Year{May 2016}
\Supervisor{Prof. George Weiss}
\Department{School of Electrical Engineering}
\Degree{Master of Science in Electrical Engineering}

\makeatother

\begin{document}

\coverpage

\titlepage

\prelimpages

\chapter*{Abstract}

Synchronous generators are an essential component of the
electric grid. Recently, the stability of the electric grid has become
an area of high interest and intensive research. One reason for that
is because the electric grid becomes more and more dependent on renewable
energy sources.

In this work we discuss the stability of a single generator connected
to an infinite bus, and show that certain reduced models fail to predict
the behavior of this system.

In the second part of this work, we discuss the stability of a grid
composed of two identical synchronous generators, and show sufficient
conditions for exponential stability. 

\tableofcontents{}

\textpages

\listoffigures

\listoftables

\printnomenclature{}

\chapter{Introduction\label{cha:introduction}}

The AC electricity grid was developed at the end of the XIX century,
and basically remained very similar till our days. Many techniques
and models that have been developed in order to analyze and design
the grid and its components are based on assumptions and methods,
which were driven by experience and observations \citep{sauerPai1998}.
In recent years, there is a trend of increasing use of renewable energy,
which requires the conversion of the energy in order to interface
with the power grid. When the rate of the converted energy will become
a significant part, it is not clear whether the traditional models
and methods for controling the power grid will succeed to control
it \citep{ZhongWeiss2011,monshizadehDePersisMonshizadehVanderSchaft2016}.
Therefore, there is increasing interest in the fundamental mathematical
models of the electrical grid.

\textit{Synchronous generator} (SG) is the main power source of the
electricity grid. The mathematical model of a SG is very complex and
difficult to use. This makes the electric grid to such a complicated
nonlinear and time-varying system that any attempt to prove its stability
analytically is hopeless. Stability analysis is usually done by simulation,
or analytically on simplified models, in which the SGs are connected
in a simple network and each SG is represented by reduced order equations,
see for instance \citep{DorflerAndBullo2012} and \citep{PorcoDorflerBullo2013}.
The reduced model of a SG is often obtained by treating the stator
currents as fast variables, thus eliminating them from the state variables
via the singular perturbation approach (see, for instance, \citep{KhalilSingularPertubations})
and keeping only the rotor angle, and the rotor angular velocity as
relevant state variables, see for instance \citep{kundur1994} and
\citep{sauerPai1998}.

SGs has the important property that once they synchronizes, means
that their rotors spin with the same velocity, they tend to remain
synchronized even without any control. This is important attribute
because the electricity grid must maintain constant frequency, and
because the ability of a SG to transfer constant power through the
grid exists only when the phase difference between each SG and the
grid phase is constants. That is the reason why it is desirable to
know if for a given grid which contains SGs and a loads,all the SGs
and tend to synchronize and if the grid frequency remains constant.
In order to use the control stability analysis, and because the trajectory
of the state of a SG in the steady state is sinusoidal, it is common
to use transformation of the voltages and currents that maps sinusoidal
trajectory into a constant point on the state plan. The famous Park's
transformation performs that, so after applying Park's transformation
on the SG model. The question whether the system is stable (which
means that all the SGs are synchronized and remain at constant frequency)
is know as frequency stability.

In this work, we study two simple grid configurations: Single SG that
connected to an infinite bus, and Two SG in parallel and a load. chapter
\ref{cha:microgrid_dynamics} will discussess the dynamical model
for the above configurations. chapter \ref{cha:equivalence_pont}
discusses the conditions to have equilibrium point at the infinite
bus configuration, and shows that micogrid comprised of two SGs can
have equilibrium point only on the synchronization manifold, and that
there is at list one such equilibrium point. chapter \ref{cha:Model_reduction}
shows that the simplified model which known as the improved swing
equation can't predict behavior that 4th order model predicts. \citep{monshizadehDePersisMonshizadehVanderSchaft2016}
showed, that the improved swing equation model predicts behavior that
the classical model (the swing equation model) can't predict. Chapter
\ref{cha:Synchronization}, shows that a microgrid that comprised
of two SGs, is stable near the manifold of synchronization.


\include{The_microgrid_dynamics}

\include{equivalence_point_existence_and_uniqueness}

\include{Model_reduction}

\include{synchronization}

\bibliographystyle{abbrvnat}
\bibliography{references}


\appendix
\include{appendix}

\newpage{}

\begin{comment}
It is possible to create the Hebrew part in \LyX{}, but this is less
of our concern. Any typesetting software like \LyX{} (or Word or
OpenOffice) is as good for this purpose. After creating the PDF file
from the Hebrew document, include it here using the Insert -> File
-> External material -> PDFpages (one of the options). See the example
below. 
\end{comment}

\end{document}
